%%%%%%%%%%%%%%%%%%%%%%%%%%%%%%%%%%%%%%%%%
% Short Sectioned Assignment
% LaTeX Template
% Version 1.0 (5/5/12)
%
% This template has been downloaded from:
% http://www.LaTeXTemplates.com
%
% Original author:
% Frits Wenneker (http://www.howtotex.com)
%
% License:
% CC BY-NC-SA 3.0 (http://creativecommons.org/licenses/by-nc-sa/3.0/)
%
%%%%%%%%%%%%%%%%%%%%%%%%%%%%%%%%%%%%%%%%%

%----------------------------------------------------------------------------------------
%	PACKAGES AND OTHER DOCUMENT CONFIGURATIONS
%----------------------------------------------------------------------------------------

\documentclass[paper=a4, fontsize=11pt]{scrartcl} % A4 paper and 11pt font size

\usepackage{listings}

\usepackage{verbatim}
\usepackage{textcomp}
\usepackage{changepage}
\usepackage[T1]{fontenc} % Use 8-bit encoding that has 256 glyphs
\usepackage{fourier} % Use the Adobe Utopia font for the document - comment this line to return to the LaTeX default
\usepackage{forest}
\usepackage[english]{babel} % English language/hyphenation
\usepackage{amsmath,amsfonts,amsthm} % Math packages

\usepackage{setspace}
\renewcommand{\baselinestretch}{1.3}
\usepackage{lipsum} % Used for inserting dummy 'Lorem ipsum' text into the template

\usepackage{color}

\usepackage{sectsty} % Allows customizing section commands
\allsectionsfont{\centering \normalfont\scshape} % Make all sections centered, the default font and small caps

\usepackage{fancyhdr} % Custom headers and footers
\pagestyle{fancyplain} % Makes all pages in the document conform to the custom headers and footers
\fancyhead{} % No page header - if you want one, create it in the same way as the footers below
\fancyfoot[L]{} % Empty left footer
\fancyfoot[C]{} % Empty center footer
\fancyfoot[R]{\thepage} % Page numbering for right footer
\renewcommand{\headrulewidth}{0pt} % Remove header underlines
\renewcommand{\footrulewidth}{0pt} % Remove footer underlines
\setlength{\headheight}{13.6pt} % Customize the height of the header

\numberwithin{equation}{section} % Number equations within sections (i.e. 1.1, 1.2, 2.1, 2.2 instead of 1, 2, 3, 4)
\numberwithin{figure}{section} % Number figures within sections (i.e. 1.1, 1.2, 2.1, 2.2 instead of 1, 2, 3, 4)
\numberwithin{table}{section} % Number tables within sections (i.e. 1.1, 1.2, 2.1, 2.2 instead of 1, 2, 3, 4)

\setlength\parindent{0pt} % Removes all indentation from paragraphs - comment this line for an assignment with lots of text

%----------------------------------------------------------------------------------------
%	TITLE SECTION
%----------------------------------------------------------------------------------------

\newcommand{\horrule}[1]{\rule{\linewidth}{#1}} % Create horizontal rule command with 1 argument of height

\title{	
\normalfont \normalsize 
\textsc{TRINITY COLLEGE DUBLIN, school of Mathematics} \\ [25pt] % Your university, school and/or department name(s)
\horrule{0.5pt} \\[0.4cm] % Thin top horizontal rule
\huge Assignment \#2, Module: MA5612 \\ % The assignment title
\horrule{2pt} \\[0.5cm] % Thick bottom horizontal rule
}

\author{Gustavo Ramirez} % Your name

\date{\normalsize\today} % Today's date or a custom date

\begin{document}

\maketitle % Print the title

%----------------------------------------------------------------------------------------
%	PROBLEM 1
%----------------------------------------------------------------------------------------

\section{Problem description}


\begin{enumerate}
\item Write a C program that multiplies two matrices. The sizes of the matrices should again be given to the program as command line arguments and be filled with values from a random number generator.
\item Use the gettimeofday() function (or some other appropriate timing routine) to measure the time taken to calculate the matrix product for various sizes of matrices. Plot a graph of your timings using gnuplot and generate a PostScript file with the graph. What conclusions, if any, can you draw about the performance of your code.
\item Play around with various compiler options for optimizing the execution of your code. Compare the performance against the unoptimized (-O0) version timed in Task 2. Which combination of flags gives the best performance?
\item (bonus marks) Read about the BLAS library and see if you can modify your code to use this library to get better performance. You will want to look at the DGEMM function.
\end{enumerate}

\newpage


\begin{comment}

USEFUL LINKS:

official sources for terminology:
-----
http://www.intel.com/content/www/us/en/support/topics/glossary.html
https://www-01.ibm.com/software/globalization/terminology/a.html
-----




about IMB processors:
-----

insert in google: list of ibm processors
https://en.wikipedia.org/wiki/List_of_IBM_products
https://www-01.ibm.com/software/passportadvantage/guide_to_identifying_processor_family.html
http://www.nextplatform.com/2015/08/10/ibm-roadmap-extends-power-chips-to-2020-and-beyond/
http://www.theverge.com/2015/7/9/8919091/ibm-7nm-transistor-processor
https://www.ibm.com/developerworks/ibmi/library/i-ibmi-7_2-and-ibm-power8/
-----




\end{comment}


\subsection{\textbf{}}

Following is the tree of files and directories for this assignment:

\begin{forest}
  for tree={
    font=\ttfamily,
    grow'=0,
    child anchor=west,
    parent anchor=south,
    anchor=west,
    calign=first,
    edge path={
      \noexpand\path [draw, \forestoption{edge}]
      (!u.south west) +(7.5pt,0) |- node[fill,inner sep=1.25pt] {} (.child anchor)\forestoption{edge label};
    },
    before typesetting nodes={
      if n=1
        {insert before={[,phantom]}}
        {}
    },
    fit=band,
    before computing xy={l=15pt},
  }
[
[assignment\_2.pdf]
[assignment\_2.tex]
  [task1
    [matmul]
    [matmul.c]
  ]
  [task2
    [exec\_times\_curve\_opt0.png]
    [exec\_times\_curve.ps]
    [execution\_times.dat]
    [gnp\_script.gp]
    [program.py]
  ]
  [task3
    [exec\_times\_curve\_opt0.png]
    [exec\_times\_curve.ps]
    [execution\_times.dat]
    [gnp\_script.gp]
    [program.py]
  ]
  [task4
    [task4.txt]
  ]
]
\end{forest}

Three parts deserve some explanation:

\begin{enumerate}
\item in the directory task1/, there is a matmul.c code where matrix multiplication is implemented. The program compiled from this code, is to be called later multiple times with the use of a Python script, to test the performance of this matrix multiplication implementation
\item in the task2/ directory, the most important file is the program.py script, from which the program task1/matmul is called multiple times, for different matrix sizes, to measure time of execution for matrix multiplication versus matrix size. After that, the gnp\_script.gp can be called with the gnuplot command to give output exec\_times\_curve.png, with a plot of time versus size (clearly, this sizes are symbolic; to understand better the range of sizes used, brief explanations are given within the program.py file, but basically, the range of matrix sizes are: $5 \times 15 \cdot 15 \times 5 \rightarrow 99 \times 15 \cdot 15 \times 99$)
\item in task3/ directory, figures can be found for different performance tests, for several used flags
\end{enumerate}


\subsection{\textbf{}}


For this part, a Python script was written for the generation of the data associated to the execution times for matrix multiplication at different sizes.

Also, a .gp file was used for the images generation with the use of Gnuplot.

For the implemented serial matrix multiplication, from the task2/exec\_times\_curve\_opt0.png file, a simple fitting gives a resulting complexity of $\mathcal{O}(n^{2})$.


\subsection{\textbf{}}


As can be seen, in the directory task3/ are seven images of different optimizations employed in the compilation of the \textit{matmul} program for matrix multiplication.

To check which flag corresponds to which image, the flag is attached at the end of the name of the file; e.g. exec\_times\_curve\_opt\_fast.png means than the -Ofast flag was used.

The same range of axis were set in all the plots, from which can be seen that the -O2 flag results in the best performance.




\end{document}