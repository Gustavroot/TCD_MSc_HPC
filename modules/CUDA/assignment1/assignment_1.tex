%%%%%%%%%%%%%%%%%%%%%%%%%%%%%%%%%%%%%%%%%
% Short Sectioned Assignment
% LaTeX Template
% Version 1.0 (5/5/12)
%
% This template has been downloaded from:
% http://www.LaTeXTemplates.com
%
% Original author:
% Frits Wenneker (http://www.howtotex.com)
%
% License:
% CC BY-NC-SA 3.0 (http://creativecommons.org/licenses/by-nc-sa/3.0/)
%
%%%%%%%%%%%%%%%%%%%%%%%%%%%%%%%%%%%%%%%%%

%----------------------------------------------------------------------------------------
%	PACKAGES AND OTHER DOCUMENT CONFIGURATIONS
%----------------------------------------------------------------------------------------

\documentclass[paper=a4, fontsize=11pt]{scrartcl} % A4 paper and 11pt font size

\usepackage{graphicx}
\graphicspath{ {images/} }
\usepackage{verbatim}
\usepackage{textcomp}
\usepackage{changepage}
\usepackage[T1]{fontenc} % Use 8-bit encoding that has 256 glyphs
\usepackage{fourier} % Use the Adobe Utopia font for the document - comment this line to return to the LaTeX default
\usepackage[english]{babel} % English language/hyphenation
\usepackage{amsmath,amsfonts,amsthm} % Math packages

\usepackage{setspace}
\renewcommand{\baselinestretch}{1.3}
\usepackage{lipsum} % Used for inserting dummy 'Lorem ipsum' text into the template

\usepackage{color}

\usepackage{sectsty} % Allows customizing section commands
\allsectionsfont{\centering \normalfont\scshape} % Make all sections centered, the default font and small caps

\usepackage{fancyhdr} % Custom headers and footers
\pagestyle{fancyplain} % Makes all pages in the document conform to the custom headers and footers
\fancyhead{} % No page header - if you want one, create it in the same way as the footers below
\fancyfoot[L]{} % Empty left footer
\fancyfoot[C]{} % Empty center footer
\fancyfoot[R]{\thepage} % Page numbering for right footer
\renewcommand{\headrulewidth}{0pt} % Remove header underlines
\renewcommand{\footrulewidth}{0pt} % Remove footer underlines
\setlength{\headheight}{13.6pt} % Customize the height of the header

\numberwithin{equation}{section} % Number equations within sections (i.e. 1.1, 1.2, 2.1, 2.2 instead of 1, 2, 3, 4)
\numberwithin{figure}{section} % Number figures within sections (i.e. 1.1, 1.2, 2.1, 2.2 instead of 1, 2, 3, 4)
\numberwithin{table}{section} % Number tables within sections (i.e. 1.1, 1.2, 2.1, 2.2 instead of 1, 2, 3, 4)

\setlength\parindent{0pt} % Removes all indentation from paragraphs - comment this line for an assignment with lots of text

%----------------------------------------------------------------------------------------
%	TITLE SECTION
%----------------------------------------------------------------------------------------

\newcommand{\horrule}[1]{\rule{\linewidth}{#1}} % Create horizontal rule command with 1 argument of height

\title{	
\normalfont \normalsize 
\textsc{TRINITY COLLEGE DUBLIN, school of Mathematics} \\ [25pt] % Your university, school and/or department name(s)
\horrule{0.5pt} \\[0.4cm] % Thin top horizontal rule
\huge Assignment \#1, Module: MA5615 \\ % The assignment title
\horrule{2pt} \\[0.5cm] % Thick bottom horizontal rule
}

\author{Gustavo Ramirez} % Your name

\date{\normalsize\today} % Today's date or a custom date

\begin{document}

\maketitle % Print the title

%----------------------------------------------------------------------------------------
%	PROBLEM 1
%----------------------------------------------------------------------------------------

\begin{comment}
\section{Problem description}

\begin{enumerate}
\item 
\item 
\item 
\item 
\end{enumerate}

\end{comment}

\newpage


\begin{comment}

USEFUL LINKS:

official sources for terminology:
-----
http://www.intel.com/content/www/us/en/support/topics/glossary.html
https://www-01.ibm.com/software/globalization/terminology/a.html
-----




about IMB processors:
-----

insert in google: list of ibm processors
https://en.wikipedia.org/wiki/List_of_IBM_products
https://www-01.ibm.com/software/passportadvantage/guide_to_identifying_processor_family.html
http://www.nextplatform.com/2015/08/10/ibm-roadmap-extends-power-chips-to-2020-and-beyond/
http://www.theverge.com/2015/7/9/8919091/ibm-7nm-transistor-processor
https://www.ibm.com/developerworks/ibmi/library/i-ibmi-7_2-and-ibm-power8/
-----




\end{comment}


\section{Part 1: serial}

After compilation, and when the program is executed (e.g. ./a.out), the calculation of 'max norm' is very fast, in comparison with the rest of methods; the other methods take, in average, around 7 or 8 times longer than max norm. These enlargements in execution time are due to jumps within the 1D array (form in which the data was stored), to read for example values of columns of the matrix, and also some operations as square root or raising to certain power, imply more execution time.

The full execution of this program is of the form: ./mat\_norms\_serial [-n N] [-m M] [-s] [-t]


\section{Part 2: parallel}

In this implementation, a use of the '-t' flag represents a considerable increase in information about timing.

As can be seen when the program is executed, a big decrease in execution time is encountered; e.g. (using 10 000 for $n$ and $m$):

\textbf{begin!}

Frobenius norm: 4096.000000

*** execution time for Frobenius norm: 0.741819

------

cudaMemcpy time: 0.063989

gpu config time: 0.000001

norm processed!

copy-back-to-host time: 0.218801

Frobenius norm with CUDA: 5773.468262

*** execution time for Frobenius norm with CUDA (excluding cudaMalloc and cuda mem copy): 0.000017

*** and including cudaMalloc and mem copy timing: 0.283485

\textbf{end!}

As can be seen, most of the time accounts for the copy from device to host, whilst the execution within the device itself is very short.

On the other hand, it's very important to note the discrepancy in the result for the norm in this case; this is due to the precision. For example, if using 100 for both dimensions of the matrix:

\textbf{begin!}

Frobenius norm: 57.409531

*** execution time for Frobenius norm: 0.000117

-------

cudaMemcpy time: 0.000033

gpu config time: 0.000000

norm processed!

copy-back-to-host time: 0.000806

Frobenius norm with CUDA: 57.409561

*** execution time for Frobenius norm with CUDA (excluding cudaMalloc and cuda mem copy): 0.000010

*** and including cudaMalloc and mem copy timing: 0.001112

\textbf{end!}

The important point about precision is that, due to the $10^{-7}$ for floats, if $n\cdot m \approx 10^6$, then the accuracies start being unreliable.

\section{Part 3: performance improvement}

A couple of automated Python scripts were developed/used for testing in this part.

The script perf\_tester.py performs all the executions (for all the matrix sizes and all the threads per block specified in the description of this assignment), then storing output results in the out.dat file. Then, the script plotter.py takes data from out\_bu.dat (to use this, simply rename out.dat to out\_bu.dat), which can be redirected to an output file or just be read from the terminal (in this case, it has been stored in the file results.txt).

As can be seen, precision becomes very important for larger matrices, and the transmission of data between RAM and global memory at the GPU, is an important aspect to be equilibrated in conjunction with how many threads to use in the calculation.

It's important to note that, from the data in part3/results.txt, 'max norm' and 'Frobenius norm' execution times become dangerously large.


\section{Part 4: double precision testing}

To make easier the transition from float to doable, the code implemented in part 2 contained the re-definition of float as VAR\_TYPE, so that the change to doable was practically trivial.

From the file part4/tests/results.txt, it's noticeable how, although the execution times increase, the values of the norms for serial and parallel are much more coincident, as for larger matrices, the error propagates at much smaller decimal positions.



\end{document}