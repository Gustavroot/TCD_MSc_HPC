%%%%%%%%%%%%%%%%%%%%%%%%%%%%%%%%%%%%%%%%%
% Short Sectioned Assignment
% LaTeX Template
% Version 1.0 (5/5/12)
%
% This template has been downloaded from:
% http://www.LaTeXTemplates.com
%
% Original author:
% Frits Wenneker (http://www.howtotex.com)
%
% License:
% CC BY-NC-SA 3.0 (http://creativecommons.org/licenses/by-nc-sa/3.0/)
%
%%%%%%%%%%%%%%%%%%%%%%%%%%%%%%%%%%%%%%%%%

%----------------------------------------------------------------------------------------
%	PACKAGES AND OTHER DOCUMENT CONFIGURATIONS
%----------------------------------------------------------------------------------------

\documentclass[paper=a4, fontsize=11pt]{scrartcl} % A4 paper and 11pt font size

\usepackage{listings}
\usepackage{verbatim}
\usepackage{textcomp}
\usepackage{changepage}
\usepackage[T1]{fontenc} % Use 8-bit encoding that has 256 glyphs
\usepackage{fourier} % Use the Adobe Utopia font for the document - comment this line to return to the LaTeX default
\usepackage[english]{babel} % English language/hyphenation
\usepackage{amsmath,amsfonts,amsthm} % Math packages

\usepackage{color}

\usepackage{setspace}
\renewcommand{\baselinestretch}{1.3}
\usepackage{lipsum} % Used for inserting dummy 'Lorem ipsum' text into the template

\usepackage{color}

\usepackage{sectsty} % Allows customizing section commands
\allsectionsfont{\centering \normalfont\scshape} % Make all sections centered, the default font and small caps

%\usepackage[usenames, dvipsnames]{color}

\usepackage{fancyhdr} % Custom headers and footers
\pagestyle{fancyplain} % Makes all pages in the document conform to the custom headers and footers
\fancyhead{} % No page header - if you want one, create it in the same way as the footers below
\fancyfoot[L]{} % Empty left footer
\fancyfoot[C]{} % Empty center footer
\fancyfoot[R]{\thepage} % Page numbering for right footer
\renewcommand{\headrulewidth}{0pt} % Remove header underlines
\renewcommand{\footrulewidth}{0pt} % Remove footer underlines
\setlength{\headheight}{13.6pt} % Customize the height of the header

\numberwithin{equation}{section} % Number equations within sections (i.e. 1.1, 1.2, 2.1, 2.2 instead of 1, 2, 3, 4)
\numberwithin{figure}{section} % Number figures within sections (i.e. 1.1, 1.2, 2.1, 2.2 instead of 1, 2, 3, 4)
\numberwithin{table}{section} % Number tables within sections (i.e. 1.1, 1.2, 2.1, 2.2 instead of 1, 2, 3, 4)

\setlength\parindent{0pt} % Removes all indentation from paragraphs - comment this line for an assignment with lots of text

%----------------------------------------------------------------------------------------
%	TITLE SECTION
%----------------------------------------------------------------------------------------

\newcommand{\horrule}[1]{\rule{\linewidth}{#1}} % Create horizontal rule command with 1 argument of height

\title{	
\normalfont \normalsize 
\textsc{TRINITY COLLEGE DUBLIN, school of Mathematics} \\ [25pt] % Your university, school and/or department name(s)
\horrule{0.5pt} \\[0.4cm] % Thin top horizontal rule
\huge Assignment \#2, Module: MA5634 \\ % The assignment title
\horrule{2pt} \\[0.5cm] % Thick bottom horizontal rule
}

\author{Gustavo Ramirez} % Your name

\date{\normalsize\today} % Today's date or a custom date

\begin{document}

\maketitle % Print the title

%----------------------------------------------------------------------------------------
%	PROBLEM 1
%----------------------------------------------------------------------------------------


\section{Poisson-distributed bank withdrawals}

\ \\
\textbf{REVIEW OF CONCEPTS:}

First, it's important to recall that a random variable is "some numerical value determined by the result of an experiment".

Both the \textit{cumulative distribution function} and the \textit{probability mass function} are defined over that random variable:

\begin{equation}
F(x) = P\{ X \leq x \}
\end{equation}

\begin{equation}
p(x) = P\{ X = x \}
\end{equation}

For a continuous random variable:

\begin{equation}
P\{ X \in C \} = \int_{C}f(x) dx
\end{equation}

where $f$ is the \textit{probability density function}.

The \textit{cumulative distribution} and the \textit{probability density} can be connected:

\begin{equation}
F(a) = P\{ X \in (-\infty, a) \} = \int_{-\infty}^{a}f(x) dx \Rightarrow \frac{d}{da}F(a) = f(a)
\end{equation}

Another perspective: $f(a)$ is the measure of how likely it is that the random variable will be near $a$.

Also, it's important to recall the expectation (or expected value) of a random variable: if $X$ is a continuous random variable having probability density function $f$, then:


\begin{equation}
E[X] = \int_{-\infty}^{\infty}xf(x)dx
\end{equation}

and also:

\begin{equation}
E[g(X)] = \int_{-\infty}^{\infty}g(x)f(x)dx
\end{equation}

Finally, let's review the concept of variance:

\begin{equation}
\mu = E[g(X)] \Rightarrow Var(X) = E[(X-\mu)^{2}]
\end{equation}


\ \\
\textbf{ON THE POISSON AND THE EXPONENTIAL DISTRIBUTIONS:}

\begin{itemize}
\item \textbf{Poisson (discrete):}

\begin{equation}
p_{i} = P\{ X = i \} = e^{-\lambda}\frac{\lambda^{i}}{i!}, \ \ \ \ i = 0, 1, ...
\label{eq:poisson1}
\end{equation}

\begin{equation}
\Rightarrow P\{ X < j \} = \sum_{i=0}^{j}p_{i}, \ \ \ \ E[X] = \lambda
\label{eq:poisson2}
\end{equation}

\item \textbf{exponential (continuous):}

\begin{equation}
f(x) = \lambda e^{-\lambda x}, \ \ \ \ 0<x<\infty
\label{eq:exponential1}
\end{equation}

\begin{equation}
\Rightarrow P\{ X < a \} = \int_{0}^{a}f(x)dx, \ \ \ \ E[X] = \frac{1}{\lambda}
\label{eq:exponential2}
\end{equation}


\end{itemize}


\ \\
\textbf{ON THE POSED PROBLEM:}

A Poisson (discrete) distribution is given for the number of withdrawls ($X$) in a single month. On the other hand, for the amount of each withdrawal ($Y$), a exponential (continuous) distribution is given.

The final goal is: \textit{calculate the probability that the total sum of withdrawals in a given month exceeds 50 000, i.e.:}

\begin{equation}
P(XY>50000)=?
\end{equation}

According to the data given in the problem (here a slight change in notation is made: $i$ refers to number of withdrawals, and $x$ refers to the amount per withdrawal):

\begin{equation}
p_{i} = P\{ X = i \} = e^{-50}\frac{50^{i}}{i!}, \ \ \ \ i = 0, 1, ...
\label{eq:poisson3}
\end{equation}

\begin{equation}
f(x) = \frac{e^{-\frac{x}{800}}}{800}, \ \ \ \ 0<x<\infty
\label{eq:exponential3}
\end{equation}



\ \\
\textit{\textbf{Superficial solution}}:

In the case of a discrete distribution, obtaining a probability is simple, as it's just $p_{i}$, while in the continuous case, $f(x)$ is a probability density, and $f(x)dx$ is the corresponding probability (of being around $x$ in an interval of width $dx$).

Then, in the specific case of study here, and for equations \ref{eq:poisson3} and \ref{eq:exponential3}, the probability  of having withdrawn $i$ times, with a $x$ amount per each withdrawal is:

\begin{equation}
P(i, x) = p_{i}f(x)dx
\label{eq:total_prob1}
\end{equation}

For each number $i$, there is a probability that $i x < 50000$ (or equivalently, that $x<\frac{50000}{i}$). Then (for fixed $i$):


\begin{equation}
P(i)|_{ix<50000, \ fixed i} = \int_{0}^{50000/i}p_{i}f(x)dx
\label{eq:total_prob2}
\end{equation}

Those probabilities have to be added, for every possibly $i$, which implies:


\begin{equation}
P_{total} = P|_{ix<50000} = p_{0}\int_{0}^{\infty}f(x)dx+\sum_{1}^{\infty}\int_{0}^{50000/i}p_{i}f(x)dx = p_{0}+\sum_{1}^{\infty}\int_{0}^{50000/i}p_{i}f(x)dx
\label{eq:total_prob3}
\end{equation}


\ \\
\textit{\textbf{Implementation of solution}}:

Finally, plugging the distributions information:

\begin{equation}
\begin{split}
P_{total} = e^{-50}+\sum_{1}^{\infty} \left( e^{-50}\frac{50^{i}}{i!} \right) \int_{0}^{50000/i}\left( \frac{e^{-\frac{x}{800}}}{800} \right)dx = e^{-50}+\sum_{1}^{\infty} \left( \left( e^{-50}\frac{50^{i}}{i!} \right) \left( 1 - e^{-50000/(800i)} \right) \right) \\
= e^{-50}+\sum_{1}^{\infty} \left( e^{-50}\frac{50^{i}}{i!} \right) - \sum_{1}^{\infty} \left( e^{-50}\frac{50^{i}}{i!} \right) e^{-50000/(800i)}  = 1 - e^{-50}\sum_{1}^{\infty} \left( e^{-62.5i}\frac{50^{i}}{i!} \right)
\end{split}
\label{eq:total_prob4}
\end{equation}


The goal is to obtain $P$ such that $ix>50000$. Then:

\begin{equation}
\begin{split}
P = 1 - P_{total}  = e^{-50}\sum_{1}^{\infty} \left( e^{-62.5i}\frac{50^{i}}{i!} \right)
\end{split}
\label{eq:total_prob4}
\end{equation}

The sum can be performed numerically, and in this case it was done with the use of a Wolfram|Alpha's online sum calculator. The result is:


\begin{equation}
\begin{split}
P (XY>50000) = e^{-50}\sum_{1}^{\infty} \left( e^{-62.5i}\frac{50^{i}}{i!} \right) = 3.59389\cdot 10^{-26}
\end{split}
\label{eq:total_prob4}
\end{equation}



\section{Monte Carlo integration}


\begin{itemize}
\item \textbf{uniform random numbers:} 





\item \textbf{exponentially distributed random numbers:} 


{\color{red} PENDING ! }


\end{itemize}


\section{Student's diet: Markov matrix}




\section{Markov Chain: predator-prey system}





\end{document}