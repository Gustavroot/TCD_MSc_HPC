%%%%%%%%%%%%%%%%%%%%%%%%%%%%%%%%%%%%%%%%%
% Short Sectioned Assignment
% LaTeX Template
% Version 1.0 (5/5/12)
%
% This template has been downloaded from:
% http://www.LaTeXTemplates.com
%
% Original author:
% Frits Wenneker (http://www.howtotex.com)
%
% License:
% CC BY-NC-SA 3.0 (http://creativecommons.org/licenses/by-nc-sa/3.0/)
%
%%%%%%%%%%%%%%%%%%%%%%%%%%%%%%%%%%%%%%%%%

%----------------------------------------------------------------------------------------
%	PACKAGES AND OTHER DOCUMENT CONFIGURATIONS
%----------------------------------------------------------------------------------------

\documentclass[paper=a4, fontsize=11pt]{scrartcl} % A4 paper and 11pt font size

\usepackage{graphicx}
\graphicspath{ {images/} }
\usepackage{verbatim}
\usepackage{textcomp}
\usepackage{changepage}
\usepackage[T1]{fontenc} % Use 8-bit encoding that has 256 glyphs
\usepackage{fourier} % Use the Adobe Utopia font for the document - comment this line to return to the LaTeX default
\usepackage[english]{babel} % English language/hyphenation
\usepackage{amsmath,amsfonts,amsthm} % Math packages

\usepackage{setspace}
\renewcommand{\baselinestretch}{1.3}
\usepackage{lipsum} % Used for inserting dummy 'Lorem ipsum' text into the template

\usepackage{color}

\usepackage{sectsty} % Allows customizing section commands
\allsectionsfont{\centering \normalfont\scshape} % Make all sections centered, the default font and small caps

\usepackage{fancyhdr} % Custom headers and footers
\pagestyle{fancyplain} % Makes all pages in the document conform to the custom headers and footers
\fancyhead{} % No page header - if you want one, create it in the same way as the footers below
\fancyfoot[L]{} % Empty left footer
\fancyfoot[C]{} % Empty center footer
\fancyfoot[R]{\thepage} % Page numbering for right footer
\renewcommand{\headrulewidth}{0pt} % Remove header underlines
\renewcommand{\footrulewidth}{0pt} % Remove footer underlines
\setlength{\headheight}{13.6pt} % Customize the height of the header

\numberwithin{equation}{section} % Number equations within sections (i.e. 1.1, 1.2, 2.1, 2.2 instead of 1, 2, 3, 4)
\numberwithin{figure}{section} % Number figures within sections (i.e. 1.1, 1.2, 2.1, 2.2 instead of 1, 2, 3, 4)
\numberwithin{table}{section} % Number tables within sections (i.e. 1.1, 1.2, 2.1, 2.2 instead of 1, 2, 3, 4)

\setlength\parindent{0pt} % Removes all indentation from paragraphs - comment this line for an assignment with lots of text

%----------------------------------------------------------------------------------------
%	TITLE SECTION
%----------------------------------------------------------------------------------------

\newcommand{\horrule}[1]{\rule{\linewidth}{#1}} % Create horizontal rule command with 1 argument of height

\title{	
\normalfont \normalsize 
\textsc{TRINITY COLLEGE DUBLIN, school of Mathematics} \\ [25pt] % Your university, school and/or department name(s)
\horrule{0.5pt} \\[0.4cm] % Thin top horizontal rule
\huge Assignment \#3, Module: MA5635 \\ % The assignment title
\horrule{2pt} \\[0.5cm] % Thick bottom horizontal rule
}

\author{Gustavo Ramirez} % Your name

\date{\normalsize\today} % Today's date or a custom date

\begin{document}

\maketitle % Print the title

%----------------------------------------------------------------------------------------
%	PROBLEM 1
%----------------------------------------------------------------------------------------

\begin{comment}
\section{Problem description}

\begin{enumerate}
\item 
\item 
\item 
\item 
\end{enumerate}

\end{comment}

\newpage





\section{Prob1}

The code for this part is in the Python script prob1/program.py.

After simulation, the resulting image for part a, i.e. multiple paths for Brownian motion up to time $t = 2$, is in directory prob1/part\_A/. Image rnd\_walk\_500\_2.png contains the result for 40 random walks.

In directory prob1/part\_B/ two histograms can be found; the first one is for $t = 1$ and the second for a value of $t = 2$.

It is important to notice that the simulations here are for the scaled random walk, which is the discretized version of Brownian motion. Ideally, when $n \rightarrow \infty$, scaled random walk tends to Brownian motion.

\section{Prob2}

Given the data from the problem description, the analytic solution to the differential equation takes the form:

$$ X(t) = X(0) \cdot \exp \left\{ \left(a - \frac{1}{2}b^{2}\right)t+b W(t) \right\} = \exp \left\{ t+W(t) \right\} \Rightarrow  X(t_{k}) = \exp \left\{ t_{k}+W(t_{k}) \right\} $$

In the directory ./prob2/ there is a Python script (program.py) that that simulates previous solution, for values $n = [4, 8, 16, 32, 64]$; the plot corresponding to that simulation can be found within that same directory (i.e. analyt\_solution\_2.png). The simulation was performed for $t = 2$.

\section{Prob3}

The Python script for the full implementation of this simulation is located in the directory ./prob3/.

In the file forward\_1.txt there are some numerical results for the error (the ones plotted). The plot of those numerical errors is displayed in the file log-log-errors.png.

The order of convergence is the slope of the curve plotted in that file. Therefore, the slope is: $\approx 4/3 \textasciitilde 1$, i.e. order of convergence is of order 1.

The error used here (and in next problem) is of the form:

$$ E = \max_{k \le n} | y(t_{k}) - t_{k} | $$


\section{Prob4}

For this part, in the Python script ./prob4/program.py, there are multiple important parts to mention: first, a random walk is generated, then, a numerical solution using Euler-Maruyama method is obtained, and finally, the analytic solution is obtained and put in a Python list, to finally compare both solutions; those three steps are performed for each value of $n$ ($ = [2, 4, 8, 16, 32]$).

The plots for both kind of errors (strong and weak), can be found in the directory ./prob4/ (as two .png files).

The order of convergence for both errors is roughly the same, approximately 15/2 ($\textasciitilde 7 $).


\end{document}