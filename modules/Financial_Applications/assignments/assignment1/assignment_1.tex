%%%%%%%%%%%%%%%%%%%%%%%%%%%%%%%%%%%%%%%%%
% Short Sectioned Assignment
% LaTeX Template
% Version 1.0 (5/5/12)
%
% This template has been downloaded from:
% http://www.LaTeXTemplates.com
%
% Original author:
% Frits Wenneker (http://www.howtotex.com)
%
% License:
% CC BY-NC-SA 3.0 (http://creativecommons.org/licenses/by-nc-sa/3.0/)
%
%%%%%%%%%%%%%%%%%%%%%%%%%%%%%%%%%%%%%%%%%

%----------------------------------------------------------------------------------------
%	PACKAGES AND OTHER DOCUMENT CONFIGURATIONS
%----------------------------------------------------------------------------------------

\documentclass[paper=a4, fontsize=11pt]{scrartcl} % A4 paper and 11pt font size

\usepackage{graphicx}
\graphicspath{ {images/} }
\usepackage{verbatim}
\usepackage{textcomp}
\usepackage{changepage}
\usepackage[T1]{fontenc} % Use 8-bit encoding that has 256 glyphs
\usepackage{fourier} % Use the Adobe Utopia font for the document - comment this line to return to the LaTeX default
\usepackage[english]{babel} % English language/hyphenation
\usepackage{amsmath,amsfonts,amsthm} % Math packages

\usepackage{setspace}
\renewcommand{\baselinestretch}{1.3}
\usepackage{lipsum} % Used for inserting dummy 'Lorem ipsum' text into the template

\usepackage{color}

\usepackage{sectsty} % Allows customizing section commands
\allsectionsfont{\centering \normalfont\scshape} % Make all sections centered, the default font and small caps

\usepackage{fancyhdr} % Custom headers and footers
\pagestyle{fancyplain} % Makes all pages in the document conform to the custom headers and footers
\fancyhead{} % No page header - if you want one, create it in the same way as the footers below
\fancyfoot[L]{} % Empty left footer
\fancyfoot[C]{} % Empty center footer
\fancyfoot[R]{\thepage} % Page numbering for right footer
\renewcommand{\headrulewidth}{0pt} % Remove header underlines
\renewcommand{\footrulewidth}{0pt} % Remove footer underlines
\setlength{\headheight}{13.6pt} % Customize the height of the header

\numberwithin{equation}{section} % Number equations within sections (i.e. 1.1, 1.2, 2.1, 2.2 instead of 1, 2, 3, 4)
\numberwithin{figure}{section} % Number figures within sections (i.e. 1.1, 1.2, 2.1, 2.2 instead of 1, 2, 3, 4)
\numberwithin{table}{section} % Number tables within sections (i.e. 1.1, 1.2, 2.1, 2.2 instead of 1, 2, 3, 4)

\setlength\parindent{0pt} % Removes all indentation from paragraphs - comment this line for an assignment with lots of text

%----------------------------------------------------------------------------------------
%	TITLE SECTION
%----------------------------------------------------------------------------------------

\newcommand{\horrule}[1]{\rule{\linewidth}{#1}} % Create horizontal rule command with 1 argument of height

\title{	
\normalfont \normalsize 
\textsc{TRINITY COLLEGE DUBLIN, school of Mathematics} \\ [25pt] % Your university, school and/or department name(s)
\horrule{0.5pt} \\[0.4cm] % Thin top horizontal rule
\huge Assignment \#1, Module: MA5635 \\ % The assignment title
\horrule{2pt} \\[0.5cm] % Thick bottom horizontal rule
}

\author{Gustavo Ramirez} % Your name

\date{\normalsize\today} % Today's date or a custom date

\begin{document}

\maketitle % Print the title

%----------------------------------------------------------------------------------------
%	PROBLEM 1
%----------------------------------------------------------------------------------------

\begin{comment}
\section{Problem description}

\begin{enumerate}
\item 
\item 
\item 
\item 
\end{enumerate}

\end{comment}

\newpage





\section{Prob1}

For a four step binomial tree, the value of a European call option at $t = 0$ is:
$$ X_{0} = \frac{1}{(1+r)^{4}}\sum_{i=0}^{4}{4 \choose i} \widetilde{p}^{i}(1-\widetilde{p})^{4-i}\max \{ S_{0}u^{i}d^{4-i} - K, 0 \} $$

where:
$$ \widetilde{p} = \frac{1+r-d}{u-d} = 0.5897 $$

and:

$$ 1- \widetilde{p} = 0.4103 $$



Also, for the amount of shares to buy, at $t = 0$ and $t = 1$, respectively:
$$ \Delta_{0} = \frac{V_{1}(H) - V_{1}(T)}{S_{1}(H) - S_{1}(T)} $$

$$ \Delta_{1}(\omega) = \frac{V_{2}(\omega H) - V_{2}(\omega T )}{S_{2}(\omega H) - S_{2}(\omega T)} \Rightarrow $$

$$ \Delta_{1}(H) = \frac{V_{2}(H H) - V_{2}(HT )}{S_{2}(H H) - S_{2}(H T)}$$

$$ \Delta_{1}(T) = \frac{V_{2}(T H) - V_{2}(T T )}{S_{2}(T H) - S_{2}(T T)}$$

with:

$$ V_{2} (HH) = \max \{ S_{0}u^{2} - K, 0 \} $$

$$ V_{2} (TH) = \max \{ S_{0}ud - K, 0 \} $$

$$ V_{2} (HT) = \max \{ S_{0}du - K, 0 \} $$

$$ V_{2} (TT) = \max \{ S_{0}d^{2} - K, 0 \} $$

and substituting values:

$$ S_{0}u^{2} = 10 \cdot 1.6^{2} = 25.6 \Rightarrow V_{2}(HH) = 14.6 $$

$$ S_{0}ud = 10 \cdot 1.6 \cdot 1.6^{-1} = 10 \Rightarrow V_{2}(HT) = V_{2}(TH) = 0 $$

$$ S_{0}d^{2} = 10 \cdot 1.6^{-2} \Rightarrow V_{2}(TT) = 0 $$

which leads to:

$$ \Delta_{1}(H) = \frac{14.6}{S_{0}(u^{2} - ud) } = 0.9359$$

$$ \Delta_{1}(T) = 0 $$

On the other hand:

$$ \Delta_{0} = \frac{V_{1}(H) - V_{1}(T)}{S_{0}(u-d)} $$

for which:

$$ V_{1}(H) = \max \{ S_{0}u - K, 0 \} = 5 $$

$$ V_{1}(T) = \max \{ S_{0}d - K, 0 \} = 0 $$

and then:

$$ \Delta_{0} = 0.5128 $$

Finally, for the call option:

$$ X_{0} = \frac{1}{(1+0.2)^{4}} \left( 0 + 0 + 0 + {4 \choose 3}\widetilde{p}^{3}(1-\widetilde{p})^{1}(S_{0}u^{2}-K) + {4 \choose 4}\widetilde{p}^{4}(S_{0}u^{4}-K) \right) =  5.55 $$



%((1)/(1+0.2)^4)(((4!)/(3!*1!))*0.5897^3*(1-0.5897)*(10*1.6^2-11) + 0.5897^4*(10*1.6^4-11) )


\newpage

\section{Prob2}

The put option is similar to call option, but when selling short, $K$ becomes an upper limit:

$$ X_{0} = \frac{1}{(1+r)^{4}}\sum_{i=0}^{4}{4 \choose i} \widetilde{p}^{i}(1-\widetilde{p})^{4-i}\max \{ K - S_{0}u^{i}d^{4-i}, 0 \} $$

and then:

$$ X_{0} = \frac{1}{(1+0.2)^{4}} \left( {4 \choose 0}(1-\widetilde{p})^{4}(K - S_{0}d^{4}) + {4 \choose 1}\widetilde{p}(1-\widetilde{p})^{3}(K - S_{0}d^{2}) + {4 \choose 2}\widetilde{p}^{2}(1-\widetilde{p})^{2}(K - S_{0}) + 0 + 0 \right) $$

$$ = 0.856 $$

%((1)/(1+0.2)^4)((1-0.5897)^4*(-10*1.6^(-4)+11) + ((4!)/(1!*3!))*(0.5897)*(1-0.5897)^3*(-10*1.6^(-2)+11) + ((4!)/(2!*2!))*(0.5897)^2*(1-0.5897)^2*(1) )



\newpage

\section{Prob3}

\subsection{$ \widetilde{E}_{1}[S_{2}](H) $}

As the obtained toss at $t = 1$ was $H$, then:

$$ S_{1}(H) = uS_{0} = 2 \cdot 4 = 8 $$

$$ \Rightarrow \widetilde{E}_{1}[S_{2}](H) = \widetilde{p}S_{2}(H) + \widetilde{q}S_{2}(T) = S_{1}(H)(\widetilde{p}u + \widetilde{q}d) = 8 \left( \frac{2}{3}2 + \frac{1}{3}\frac{1}{2} \right) = 12 $$

\subsection{$ \widetilde{E}_{1}[S_{2}](T) $}

$$ S_{1}(T) = dS_{0} = 2^{-1} \cdot 4 = 2 $$

$$ \Rightarrow \widetilde{E}_{1}[S_{2}](T) = \widetilde{p}S_{2}(H) + \widetilde{q}S_{2}(T) = S_{1}(T)(\widetilde{p}u + \widetilde{q}d) = 2 \left( \frac{2}{3}2 + \frac{1}{3}\frac{1}{2} \right) = 3 $$


\subsection{$ \widetilde{E}_{1}[S_{3}](T) $}


$$ S_{1}(T) = dS_{0} = 2^{-1} \cdot 4 = 2 $$

$$ \Rightarrow \widetilde{E}_{1}[S_{3}](T) = \left( 8 \cdot \widetilde{p}^{2} + 2 (\cdot 2 \cdot \widetilde{p}\widetilde{q}) + \frac{1}{2}\widetilde{q}^{2} \right) = 4.5 $$



\newpage

\section{Prob4}

\subsection{}

\textbf{FIRST:}

$$ E_{1}[S_{2}+S_{3}](H) = ( p^{2}(16+32) + pq(16+8) + pq(4+8) + q^{2}(4+2) ) = 30 $$

% ( 48*(2/3)^2 + (2/3)*(1/3)*24 + (2/3)*(1/3)*12 + 6*(1/3)^2 )

% 8( (2/3)*2 + (1/3)*(1/2) )

$$ E_{1}[S_{2}](H) = 12 $$

$$ E_{1}[S_{3}](H) = 32\left(\frac{2}{3}\right)^2 + 2 \cdot 8 \cdot (2/3) \cdot (1/3) + 2 \cdot (1/3)^2 = 18 $$

% 32*(2/3)^2 + 2*8*(2/3)*(1/3) + 2*(1/3)^2

$$ \Rightarrow E_{1}[S_{2} + S_{3}](H) = E_{1}[S_{2}](H) + E_{1}[S_{3}](H) $$


\textbf{SECOND:}

$$ E_{1}[S_{2}+S_{3}](T) = p^{2}(4+8) + pq(4+2) + pq(1+2) + q^{2}(1+0.5) = 7.5 $$

% ( 12*(2/3)^2 + (2/3)*(1/3)*6 + (2/3)*(1/3)*3 + 1.5*(1/3)^2 )

$$ E_{1}[S_{2}](T) = 3 $$

$$ E_{1}[S_{3}](T) = 4.5 $$

$$ \Rightarrow E_{1}[S_{2} + S_{3}](T) = E_{1}[S_{2}](T) + E_{1}[S_{3}](T) $$



\subsection{}

\textbf{FIRST PART:}

$$ E_{1}[S_{1}S_{2}](H) = p(8 \cdot 16) + q(8 \cdot 4) = 94 $$

% ( 16*8*(2/3) + 4*8*(1/3) )

$$ E_{1}[S_{2}](H) = 12 $$

$$ S_{1}(H) = 8 $$

$$ E_{1}[S_{2}](H) \cdot S_{1}(H) = 96 = E_{1}[S_{1}S_{2}](H) $$



\textbf{SECOND PART:}

$$ E_{1}[S_{1}S_{2}](T) = p(2 \cdot 4) + q(2 \cdot 1) = 6 $$

% ( 2*4*(2/3) + 2*1*(1/3) )

$$ E_{1}[S_{2}](T) = 3 $$

$$ S_{1}(T) = 2 $$

$$ E_{1}[S_{2}](T) \cdot S_{1}(T) = 6 = E_{1}[S_{1}S_{2}](T) $$




\subsection{}


First:

$$ E_{2}[S_{3}](HH) = 32p + 8q = 24 $$

% 32*(2/3) + 8*(1/3)

$$ E_{2}[S_{3}](HT) = 8p + 2q = 6 $$

% 8*(2/3) + 2*(1/3)

$$ E_{2}[S_{3}](TH) = E_{2}[S_{3}](HT) = 6 $$

$$ E_{2}[S_{3}](TT) = 2p + 0.5q = 1.5 $$

% 2*(2/3) + 0.5*(1/3)





then:


$$ E_{1}[E_{2}[S_{3}]](H) = pE_{2}[S_{3}](HH) + qE_{2}[S_{3}](HT) = 18 $$

% (2/3)*24 + (1/3)*6

$$ E_{1}[E_{2}[S_{3}]](H) = pE_{2}[S_{3}](TH) + qE_{2}[S_{3}](TT) = 4.5 $$

% (2/3)*6 + (1/3)*1.5




finally:

$$ E_{1}S_{3}(H) = 18 = E_{1}[E_{2}[S_{3}]](H) $$

$$ E_{1}S_{3}(T) = 4.5 = E_{1}[E_{2}[S_{3}]](T) $$








\newpage

\section{Prob5}

\subsection{}

$$ E_{n}[M_{n+1}] = E_{n}[S_{n+1}] = pS_{n+1}(H) + qS_{n+1}(T) = S_{n}(pu + qd) = S_{n}((1/3)\cdot 2 + (2/3)\cdot (1/2)) = S_{n} $$

Therefore, as $ M_{n} = E_{n}[M_{n+1}] $, $M_{n}$ is a martingale in this case.



\subsection{}

$$ E_{n}[M_{n+1}] = \frac{1}{(1+r)^{n+1}}E_{n}[S_{n+1}] = \frac{1}{(1+r)^{n+1}}(pS_{n+1}(H) + qS_{n+1}(T)) =  $$

$$ \frac{1}{(1+r)^{n}}\frac{1}{1+r}(S_{n}(pu + qd)) = M_{n}\frac{pu+qd}{1+r} = M_{n}\frac{(1/2)(2)+(1/2)(1/2)}{1+1/4} = M_{n} $$

Therefore, as $ M_{n} = E_{n}[M_{n+1}] $, $M_{n}$ is a martingale in this case.

\subsection{}

$$ E_{n}[M_{n+1}] = E_{n}[S_{n+1}] = pS_{n+1}(H) + qS_{n+1}(T) = S_{n}(pu + qd) = S_{n}((1/2)\cdot 2 + (1/2)\cdot (1/2)) = \frac{5}{4}S_{n} $$

Therefore, as $ M_{n} \neq E_{n}[M_{n+1}] $, $M_{n}$ is not a martingale in this case.



\subsection{}

\textbf{FIRST:}

$$ E_{n}[M_{n+1}] = E_{n}[S_{n+1}] = pS_{n+1}(H) + qS_{n+1}(T) = S_{n}(pu + qd) = S_{n}((2/3)\cdot 2 + (1/3)\cdot (1/2)) = \frac{3}{2}S_{n} $$

Therefore, as $ M_{n} \neq E_{n}[M_{n+1}] $, $M_{n}$ is not a martingale in this case.


\textbf{SECOND:}

$$ E_{n}[M_{n+1}] = \frac{1}{(1+r)^{n+1}}E_{n}[S_{n+1}] = \frac{1}{(1+r)^{n+1}}(pS_{n+1}(H) + qS_{n+1}(T)) =  $$

$$ \frac{1}{(1+r)^{n}}\frac{1}{1+r}(S_{n}(pu + qd)) = M_{n}\frac{pu+qd}{1+r} = M_{n}\frac{(2/3)(2)+(1/3)(1/2)}{1+1/4} = 1.2 \cdot M_{n} $$

Therefore, as $ M_{n} \neq E_{n}[M_{n+1}] $, $M_{n}$ is not a martingale in this case.




\subsection{}


In this case, a risk-neutral measure pair $(p, q)$ is one such that:

$$ \frac{pu+qd}{1+r} = 1 \Rightarrow pu+qd = \frac{5}{4} \Rightarrow p = \frac{1}{2} \left( \frac{5}{4}-\frac{1}{2}q \right) $$

and in all those previous given cases, only $\left(  \frac{1}{2}, \frac{1}{2} \right) $ was a risk-neutral measure.



\section{Prob6}

The risk-neutral probabilities given are:

$$ \widetilde{p} = \frac{1+r-d}{u-d} = \frac{1+1/4-1/2}{2-1/2} = \frac{1}{2} $$

$$ \widetilde{p} = \frac{u-(1+r)}{u-d} = \frac{2-(1+1/4)}{2-1/2} = \frac{1}{2} $$

% (2-(1+1/4))/(2-1/2)

Therefore, with:

$$ M_{n} = \frac{S_{n}}{(1+r)^{n}} $$

then:

$$ \widetilde{E}_{n}[M_{n+1}] = \frac{1}{(1+r)^{n}}\frac{1}{1+r}S_{n}(\widetilde{p}u + \widetilde{q}d) = \frac{S_{n}}{(1+r)^{n}}\frac{\widetilde{p}u + \widetilde{q}d}{1+r} = \frac{S_{n}}{(1+r)^{n}}\frac{(1/2)2 + (1/2)(1/2)}{1+r} = \frac{S_{n}}{(1+r)^{n}} = M_{n} $$

Therefore, as $ M_{n} = E_{n}[M_{n+1}] $, $M_{n}$ is a martingale under those risk-neutral probabilities.


\newpage

\section{Prob7}

Implemented using Python.

The code for part \textbf{a} can be found at ./prob7/program.py (IMPORTANT: the code for ALL parts of this assignment is in that file).

The plots resulting of the implementations for part \textbf{b} are in ./prob7/part\_B/.

The plot for part \textbf{c} is in ./prob7/part\_C/.

\subsection{Part D}

For part \textbf{d}, the resulting plot can be found in ./prob7/part\_D/.

The distribution towards which these final values arrays should approximate, when plotted as a histogram, is a normal distribution with mean 0 and standard deviation $t$. As can be seen from the plots (both located at ./prob7/part\_D/), the histogram resembles the distribution just mentioned (for both values of $t$ used).












\end{document}