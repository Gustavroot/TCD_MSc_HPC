%%%%%%%%%%%%%%%%%%%%%%%%%%%%%%%%%%%%%%%%%
% Short Sectioned Assignment
% LaTeX Template
% Version 1.0 (5/5/12)
%
% This template has been downloaded from:
% http://www.LaTeXTemplates.com
%
% Original author:
% Frits Wenneker (http://www.howtotex.com)
%
% License:
% CC BY-NC-SA 3.0 (http://creativecommons.org/licenses/by-nc-sa/3.0/)
%
%%%%%%%%%%%%%%%%%%%%%%%%%%%%%%%%%%%%%%%%%

%----------------------------------------------------------------------------------------
%	PACKAGES AND OTHER DOCUMENT CONFIGURATIONS
%----------------------------------------------------------------------------------------

\documentclass[paper=a4, fontsize=11pt]{scrartcl} % A4 paper and 11pt font size

\usepackage{graphicx}
\graphicspath{ {images/} }
\usepackage{verbatim}
\usepackage{textcomp}
\usepackage{changepage}
\usepackage[T1]{fontenc} % Use 8-bit encoding that has 256 glyphs
\usepackage{fourier} % Use the Adobe Utopia font for the document - comment this line to return to the LaTeX default
\usepackage[english]{babel} % English language/hyphenation
\usepackage{amsmath,amsfonts,amsthm} % Math packages

\usepackage{setspace}
\renewcommand{\baselinestretch}{1.3}
\usepackage{lipsum} % Used for inserting dummy 'Lorem ipsum' text into the template

\usepackage{color}

\usepackage{sectsty} % Allows customizing section commands
\allsectionsfont{\centering \normalfont\scshape} % Make all sections centered, the default font and small caps

\usepackage{fancyhdr} % Custom headers and footers
\pagestyle{fancyplain} % Makes all pages in the document conform to the custom headers and footers
\fancyhead{} % No page header - if you want one, create it in the same way as the footers below
\fancyfoot[L]{} % Empty left footer
\fancyfoot[C]{} % Empty center footer
\fancyfoot[R]{\thepage} % Page numbering for right footer
\renewcommand{\headrulewidth}{0pt} % Remove header underlines
\renewcommand{\footrulewidth}{0pt} % Remove footer underlines
\setlength{\headheight}{13.6pt} % Customize the height of the header

\numberwithin{equation}{section} % Number equations within sections (i.e. 1.1, 1.2, 2.1, 2.2 instead of 1, 2, 3, 4)
\numberwithin{figure}{section} % Number figures within sections (i.e. 1.1, 1.2, 2.1, 2.2 instead of 1, 2, 3, 4)
\numberwithin{table}{section} % Number tables within sections (i.e. 1.1, 1.2, 2.1, 2.2 instead of 1, 2, 3, 4)

\setlength\parindent{0pt} % Removes all indentation from paragraphs - comment this line for an assignment with lots of text

%----------------------------------------------------------------------------------------
%	TITLE SECTION
%----------------------------------------------------------------------------------------

\newcommand{\horrule}[1]{\rule{\linewidth}{#1}} % Create horizontal rule command with 1 argument of height

\title{	
\normalfont \normalsize 
\textsc{TRINITY COLLEGE DUBLIN, school of Mathematics} \\ [25pt] % Your university, school and/or department name(s)
\horrule{0.5pt} \\[0.4cm] % Thin top horizontal rule
\huge Assignment \#4, Module: MA5611 \\ % The assignment title
\horrule{2pt} \\[0.5cm] % Thick bottom horizontal rule
}

\author{Gustavo Ramirez} % Your name

\date{\normalsize\today} % Today's date or a custom date

\begin{document}

\maketitle % Print the title

%----------------------------------------------------------------------------------------
%	PROBLEM 1
%----------------------------------------------------------------------------------------

\begin{comment}
\section{Problem description}

\begin{enumerate}
\item 
\item 
\item 
\item 
\end{enumerate}

\end{comment}

\newpage


\begin{comment}

USEFUL LINKS:

official sources for terminology:
-----
http://www.intel.com/content/www/us/en/support/topics/glossary.html
https://www-01.ibm.com/software/globalization/terminology/a.html
-----




about IMB processors:
-----

insert in google: list of ibm processors
https://en.wikipedia.org/wiki/List_of_IBM_products
https://www-01.ibm.com/software/passportadvantage/guide_to_identifying_processor_family.html
http://www.nextplatform.com/2015/08/10/ibm-roadmap-extends-power-chips-to-2020-and-beyond/
http://www.theverge.com/2015/7/9/8919091/ibm-7nm-transistor-processor
https://www.ibm.com/developerworks/ibmi/library/i-ibmi-7_2-and-ibm-power8/
-----




\end{comment}


\section{Part 1: serial}


Serial implementations can be found in directories task1/ and task2/.

\textbf{Gauss:}

In task1/, gauss.c is an implementation of Gauss elimination method. That implementation uses random numbers to set the augmented matrix, so there are no problems with zeroes on the diagonal, and therefore no need for pivoting; for this reason, the implementation there is a very simple without partial pivoting one.

\textbf{Sieve:}

In task2/, sieve.c is a serial implementation of Sieve of Eratosthenes. Given the max number N, it's based on a creation of a list of numbers from 0 to N, and then marking which of those is a prime and which isn't. Due to the double loop implementation, it's suitable for OpenMP pragmas use.



\section{Part 2: parallel}


Parallel implementations can be found in directory task3/.

In the case of Gauss elimination, the OpenMP pragma is used to optimize the creation of the upper triangular part. Specifically, on each column, the parallelization is implemented by dividing the rows statically over the threads.

On the other hand, when implementing Sieve of Eratosthenes, because of the non-uniform increase on the tested numbers, an OpenMP dynamic implementation is appropriate.


\section{Part 3: processing scripts and results}

The script data\_gen.py executes the serial and parallel implementations for multiple values of n (both for Gauss elimination and Sieve of Eratosthenes); also, makes the executions for 2, 4 and 8 cores. The data results (output of this Python script) can be found on the directory ./results/.

Finally, the Python script plotter.py reads all the .dat files in ./results/, plotting the results and locating the plots in the directory ./plots/.

There's a speedup in both cases after implementing parallel OpenMP pragmas.


\end{document}