%%%%%%%%%%%%%%%%%%%%%%%%%%%%%%%%%%%%%%%%%
% Short Sectioned Assignment
% LaTeX Template
% Version 1.0 (5/5/12)
%
% This template has been downloaded from:
% http://www.LaTeXTemplates.com
%
% Original author:
% Frits Wenneker (http://www.howtotex.com)
%
% License:
% CC BY-NC-SA 3.0 (http://creativecommons.org/licenses/by-nc-sa/3.0/)
%
%%%%%%%%%%%%%%%%%%%%%%%%%%%%%%%%%%%%%%%%%

%----------------------------------------------------------------------------------------
%	PACKAGES AND OTHER DOCUMENT CONFIGURATIONS
%----------------------------------------------------------------------------------------

\documentclass[paper=a4, fontsize=11pt]{scrartcl} % A4 paper and 11pt font size

\usepackage{graphicx}
\graphicspath{ {images/} }
\usepackage{verbatim}
\usepackage{textcomp}
\usepackage{changepage}
\usepackage[T1]{fontenc} % Use 8-bit encoding that has 256 glyphs
\usepackage{fourier} % Use the Adobe Utopia font for the document - comment this line to return to the LaTeX default
\usepackage[english]{babel} % English language/hyphenation
\usepackage{amsmath,amsfonts,amsthm} % Math packages

\usepackage{setspace}
\renewcommand{\baselinestretch}{1.3}
\usepackage{lipsum} % Used for inserting dummy 'Lorem ipsum' text into the template

\usepackage{color}

\usepackage{sectsty} % Allows customizing section commands
\allsectionsfont{\centering \normalfont\scshape} % Make all sections centered, the default font and small caps

\usepackage{fancyhdr} % Custom headers and footers
\pagestyle{fancyplain} % Makes all pages in the document conform to the custom headers and footers
\fancyhead{} % No page header - if you want one, create it in the same way as the footers below
\fancyfoot[L]{} % Empty left footer
\fancyfoot[C]{} % Empty center footer
\fancyfoot[R]{\thepage} % Page numbering for right footer
\renewcommand{\headrulewidth}{0pt} % Remove header underlines
\renewcommand{\footrulewidth}{0pt} % Remove footer underlines
\setlength{\headheight}{13.6pt} % Customize the height of the header

\numberwithin{equation}{section} % Number equations within sections (i.e. 1.1, 1.2, 2.1, 2.2 instead of 1, 2, 3, 4)
\numberwithin{figure}{section} % Number figures within sections (i.e. 1.1, 1.2, 2.1, 2.2 instead of 1, 2, 3, 4)
\numberwithin{table}{section} % Number tables within sections (i.e. 1.1, 1.2, 2.1, 2.2 instead of 1, 2, 3, 4)

\setlength\parindent{0pt} % Removes all indentation from paragraphs - comment this line for an assignment with lots of text

%----------------------------------------------------------------------------------------
%	TITLE SECTION
%----------------------------------------------------------------------------------------

\newcommand{\horrule}[1]{\rule{\linewidth}{#1}} % Create horizontal rule command with 1 argument of height

\title{	
\normalfont \normalsize 
\textsc{TRINITY COLLEGE DUBLIN, school of Mathematics} \\ [25pt] % Your university, school and/or department name(s)
\horrule{0.5pt} \\[0.4cm] % Thin top horizontal rule
\huge Assignment \#3, Module: MA5611 \\ % The assignment title
\horrule{2pt} \\[0.5cm] % Thick bottom horizontal rule
}

\author{Gustavo Ramirez} % Your name

\date{\normalsize\today} % Today's date or a custom date

\begin{document}

\maketitle % Print the title

%----------------------------------------------------------------------------------------
%	PROBLEM 1
%----------------------------------------------------------------------------------------

\begin{comment}
\section{Problem description}

\begin{enumerate}
\item 
\item 
\item 
\item 
\end{enumerate}

\end{comment}

\newpage


\begin{comment}

USEFUL LINKS:

official sources for terminology:
-----
http://www.intel.com/content/www/us/en/support/topics/glossary.html
https://www-01.ibm.com/software/globalization/terminology/a.html
-----




about IMB processors:
-----

insert in google: list of ibm processors
https://en.wikipedia.org/wiki/List_of_IBM_products
https://www-01.ibm.com/software/passportadvantage/guide_to_identifying_processor_family.html
http://www.nextplatform.com/2015/08/10/ibm-roadmap-extends-power-chips-to-2020-and-beyond/
http://www.theverge.com/2015/7/9/8919091/ibm-7nm-transistor-processor
https://www.ibm.com/developerworks/ibmi/library/i-ibmi-7_2-and-ibm-power8/
-----




\end{comment}


\section{Part 1}




\section{Part 2}

From the fitting of the execution times (see the ./task2/fit.log file), the following values are obtained:

$a = 1.39018 \cdot 10^{-11}$

$b = 2.59096$

$c = 0.965511$

for the following form of the exponential:

f(x)=a*exp(x*b)+c

$$ f(x) = a\cdot e^{x\cdot b} + c $$

and therefore:

$$ f(x) = 1.39018 \cdot 10^{-11} \cdot e^{x\cdot 2.6} + 0.965511 $$

from which the times for execution are, approximately (in seconds and days):

$$ T(20) \textasciitilde 532579070245 \ seconds \textasciitilde 16887.97 \ years $$

$$ T(50) \textasciitilde 4 \cdot 10^{45} \ seconds \textasciitilde 10^{38} \ years $$

$$ T(20) \textasciitilde 114 \cdot 10^{100} \ seconds \textasciitilde 4 \cdot 10^{94} \ years $$

\section{Part 3}













\end{document}