%%%%%%%%%%%%%%%%%%%%%%%%%%%%%%%%%%%%%%%%%
% Short Sectioned Assignment
% LaTeX Template
% Version 1.0 (5/5/12)
%
% This template has been downloaded from:
% http://www.LaTeXTemplates.com
%
% Original author:
% Frits Wenneker (http://www.howtotex.com)
%
% License:
% CC BY-NC-SA 3.0 (http://creativecommons.org/licenses/by-nc-sa/3.0/)
%
%%%%%%%%%%%%%%%%%%%%%%%%%%%%%%%%%%%%%%%%%

%----------------------------------------------------------------------------------------
%	PACKAGES AND OTHER DOCUMENT CONFIGURATIONS
%----------------------------------------------------------------------------------------

\documentclass[paper=a4, fontsize=11pt]{scrartcl} % A4 paper and 11pt font size

\usepackage{listings}

\usepackage{verbatim}
\usepackage{textcomp}
\usepackage{changepage}
\usepackage[T1]{fontenc} % Use 8-bit encoding that has 256 glyphs
\usepackage{fourier} % Use the Adobe Utopia font for the document - comment this line to return to the LaTeX default
\usepackage{forest}
\usepackage[english]{babel} % English language/hyphenation
\usepackage{amsmath,amsfonts,amsthm} % Math packages

\usepackage{setspace}
\renewcommand{\baselinestretch}{1.3}
\usepackage{lipsum} % Used for inserting dummy 'Lorem ipsum' text into the template

\usepackage{color}

\usepackage{sectsty} % Allows customizing section commands
\allsectionsfont{\centering \normalfont\scshape} % Make all sections centered, the default font and small caps

\usepackage{fancyhdr} % Custom headers and footers
\pagestyle{fancyplain} % Makes all pages in the document conform to the custom headers and footers
\fancyhead{} % No page header - if you want one, create it in the same way as the footers below
\fancyfoot[L]{} % Empty left footer
\fancyfoot[C]{} % Empty center footer
\fancyfoot[R]{\thepage} % Page numbering for right footer
\renewcommand{\headrulewidth}{0pt} % Remove header underlines
\renewcommand{\footrulewidth}{0pt} % Remove footer underlines
\setlength{\headheight}{13.6pt} % Customize the height of the header

\numberwithin{equation}{section} % Number equations within sections (i.e. 1.1, 1.2, 2.1, 2.2 instead of 1, 2, 3, 4)
\numberwithin{figure}{section} % Number figures within sections (i.e. 1.1, 1.2, 2.1, 2.2 instead of 1, 2, 3, 4)
\numberwithin{table}{section} % Number tables within sections (i.e. 1.1, 1.2, 2.1, 2.2 instead of 1, 2, 3, 4)

\setlength\parindent{0pt} % Removes all indentation from paragraphs - comment this line for an assignment with lots of text

%----------------------------------------------------------------------------------------
%	TITLE SECTION
%----------------------------------------------------------------------------------------

\newcommand{\horrule}[1]{\rule{\linewidth}{#1}} % Create horizontal rule command with 1 argument of height

\title{	
\normalfont \normalsize 
\textsc{TRINITY COLLEGE DUBLIN, school of Mathematics} \\ [25pt] % Your university, school and/or department name(s)
\horrule{0.5pt} \\[0.4cm] % Thin top horizontal rule
\huge Assignment \#3, Module: MA5633 \\ % The assignment title
\horrule{2pt} \\[0.5cm] % Thick bottom horizontal rule
}

\author{Gustavo Ramirez} % Your name

\date{\normalsize\today} % Today's date or a custom date

\begin{document}

\maketitle % Print the title

%----------------------------------------------------------------------------------------
%	PROBLEM 1
%----------------------------------------------------------------------------------------



In this document are described the algorithms employed for solving the different tasks of this assignment.



\newpage


\begin{comment}

USEFUL LINKS:

official sources for terminology:
-----
http://www.intel.com/content/www/us/en/support/topics/glossary.html
https://www-01.ibm.com/software/globalization/terminology/a.html
-----




about IMB processors:
-----

insert in google: list of ibm processors
https://en.wikipedia.org/wiki/List_of_IBM_products
https://www-01.ibm.com/software/passportadvantage/guide_to_identifying_processor_family.html
http://www.nextplatform.com/2015/08/10/ibm-roadmap-extends-power-chips-to-2020-and-beyond/
http://www.theverge.com/2015/7/9/8919091/ibm-7nm-transistor-processor
https://www.ibm.com/developerworks/ibmi/library/i-ibmi-7_2-and-ibm-power8/
-----




\end{comment}


\section{LU factorisation with partial pivoting}

When preparing a matrix $A$ to solve a system of equations of the form $A\vec{x} = \vec{b}$, some entries in the matrix can have a much different scale than the rest of the values in that matrix, which may lead to "divisions by zero" (due to the fact that the value is indeed zero, or due to the resolution of values stored in the computer, which makes it zero). The more prone to happen this in an algorithm, the more \textit{unstable} is that algorithm.

A way to make certain algorithm more stable, is adding to it an extra step: \textit{pivoting}. There are three types of pivoting: partial, complete and scaling. We will focus here on partial pivoting.

The algorithm to be used here is \textit{LU decomposition}:

\begin{center}
\textit{if a system of equations of the form $A\vec{x} = \vec{b}$ is to be solved, then the matrix $A$ can be decomposed in the following way: $A = LU$, such that: $A\vec{x} = (LU)\vec{x} = \vec{b}$, from which we can find $\vec{x}$ by first solving $L\vec{y} = \vec{b}$ for $\vec{y}$, and then solving $U\vec{x} = \vec{y}$ for $\vec{x}$}
\end{center}

When applying that LU decomposition algorithm naively, i.e. by using the following equations:

\begin{equation}
u_{11} = a_{11}, \ u_{12} = a_{12},..., \ u_{1n} = a_{1n}, \ l_{ii} = 1
\end{equation}

\begin{equation}
u_{ij} = a_{ij}-\sum_{k=1}^{i-1}u_{kj}l_{ik}, \ i>1
\end{equation}

\begin{equation}
l_{ij} = \frac{1}{u_{jj}} \left( a_{ij}-\sum_{k=1}^{j-1}u_{kj}l_{ik} \right)
\end{equation}

If any of those $u_{jj}$ is zero, the system becomes unstable; the solution to this is \textit{partial pivoting}: rearrange the rows of $A$ in a way that, in each column, the largest element is at the diagonal.

When pivoting, each time a permutation of rows is performed, this represents a permutation in the vector $\vec{b}$. Permutations can be written as a matrix acting over $A$: $A_{p} = PA$, where $A_{p}$ is a matrix where the diagonal entries are the largest on their corresponding columns; this implies:

\begin{equation}
A_{p} =L(U\vec{x})= (PA)\vec{x} = P(A\vec{x}) = P(\vec{b})
\end{equation}

i.e. $P$ is obtained previous to the LU decomposition, the decomposition is applied to $A_{p}$, and then the solution $\vec{x}$ is obtained in two steps:

\begin{equation}
L\vec{y} = P\vec{b} \Rightarrow U\vec{x} = \vec{y}
\end{equation}

Once the LU decomposition has taken place, the solutions are obtained as:

\begin{equation}
y_{i} = \left( (P\vec{b})_{i}-\sum_{j=0}^{i-1}l_{ij}y_{j} \right)\frac{1}{l_{ii}}
\end{equation}

for $i$ going from $i=0$ to $i=n-1$, and finally:

\begin{equation}
x_{i} = \left( y_{i} - \sum_{j=i+1}^{n-1}u_{ij}x_{j} \right) \frac{1}{u_{ii}}
\end{equation}

for $i$ going from $i=n-1$ to $i=0$.



\section{Condition number estimator}

To study the condition number estimator, it's necessary to introduce the infinity norm of a matrix:

\begin{equation}
||A||_{\infty} = \max_{1\le i \le n}\left( \sum_{j=1}^{n}|a_{ij}| \right)
\end{equation}

and in the case of vectors:

\begin{equation}
||x||_{\infty} = \max_{1 \le i \le n}|x_{i}|
\end{equation}


The condition number gives some limits for the variations of the solutions of the linear system of equations, in terms of variations on their data. The bigger the condition number, the more prone to fluctuate is the solution when small changes on the data of the system are made.


\section{Jacobi and Gauss-Seidel iterative solvers}

\subsection{Jacobi}

The algorithm is very simple: from the system of equation of the form $A\vec{x} = \vec{b}$, from the matrix $A$ is extracted only the diagonal: $D_{ii} = A_{ii}$, $D_{ij} = 0, \ i\ne j$, and then another matrix is created: $R = A-D$.

Then, the solution of the system is obtained through the use of the iterative formula:

\begin{equation}
x^{k+1} = D^{-1}(n-Rx^{k})
\end{equation}

starting with an initial guess $x^{0}$, and then iterating.

\subsection{Gauss-Seidel}

Gauss-Seidel method is similar to Jacobi's Method, both being iterative methods for solving systems of linear equations, but Gauss-Seidel converges somewhat quicker in serial applications.

The method consists in the following: if the matrix $A$ is decomposed into $L$ (this lower matrix containing the diagonal as well) and $U$ matrices, and after that split, the idea is to apply the following iterative formula:

\begin{equation}
x^{k+1} = L^{-1}(b-Ux^{k})
\end{equation}


\section{Forward Euler's method}



\section{Trapezoid rule predictor-corrector method}



\section{Second order Runge-Kutta}



\section{Fourth order Runge-Kutta}










\end{document}